\section{Fazit}

Die genauere theoretische Betrachtung, wie auch die Ergebnisse des Benchmarks zeigen deutlich, dass unser Ansatz nicht gut geeignet ist für eine Implementierung auf der GPU.
Die CPU hat, vermutlich durch den größeren und schnelleren Cache, Vorteile bei den zufälligen Zugriffen auf die Daten.
Auch die höhere Anzahl an Kernen konnte der GPU keinen wirklichen Vorteil verschaffen, obwohl eine hohe Anzahl an Ameisen also an Kernels eingesetzt werden.

Bei einem zweiten HPC Projekt könnte man früher im Prozess ein Struktogramm ausarbeiten, bzw. mögliche Speicherzugriffe vorhersagen und somit früher erkennen, dass das Problem eher für die CPU gemacht ist.
Trotzdem zeigt dieser Anwendungsfall deutlich, die Unterschiede zwischen den beiden Komponenten, besonders wenn man die OpenCL Lösung auf der CPU ausführt.

