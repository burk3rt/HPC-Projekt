\section{Motivation 0,5 S}
Das Travelling-Salesman-Problem (\acs*{tsp}) ist ein Optimierungsproblem bzw. ein Rundreiseproblem aus dem Operations Reasearch Umfeld.
\\\\
Ein Reisender muss eine begrenzte Anzahl an Städten besuchen und sucht nach einer optimalen Strecke, die alle Städte genau einmal enthält. 
Eine optimale Lösung ist dabei diejenige, die am kürzesten, günstigsten oder schnellsten ist, je nachdem, welches Ziel man verfolgt. \cite*{TravellingSalesmanProblemProblemHandlungsreisenden}
\\\\
Das TSP findet in vielen verschiedenen Bereichen Anwendung, in denen es darum geht, Routen oder Touren zu planen, bei denen alle Ziele besucht werden müssen. \\
Beispiele dafür sind:
\begin{itemize}
    \item Produktionsprozesse:
            Einkauf, Produktion, Lagerung und Belieferung der Kunden möglichst kostengünstig unter Einhaltung von Fristen und Kapazitäten.
    \item Militär:
            Eine Truppe muss alle Stützpunkte besuchen und dabei möglichst wenig Ressourcen verbrauchen. 
    \item Tourismus:
            Ein Tourist möchte möglichst viele Sehenswürdigkeiten in einer Stadt besuchen und möglichst die Route mit der kürzesten Strecke nehmen.
\end{itemize}

\noindent Unser Ziel ist es das Tourismus-Problem zu lösen und die Laufzeit des Algorithmus zur Berechnung der Strecke zu optimieren.
Dabei betrachten wird ausschließlich symmetrische Graphen.

\section{Problemstellung 0,5 - 1 S}
Das \acs*{tsp} ist ein NP-schweres Problem, was bedeutet, dass es schwierig ist, eine effiziente Lösung zu finden. 
Die Anzahl der verschiedenen möglichen Routen hängt davon ab, ob es ein Symmetrischer oder Asymmetrischer Graph ist.\cite*{gmbhTravelingSalesmanProblemOperationsResearch}

\begin{itemize}
    \item Symmetrischer Graph: Kantenlängen von einem zum anderen Konten sind in beiden Richtungen identisch.\\
            Anzahl Routen: (n-1)!/2\\
            Bsp.:   15 Städte: (15-1)!/2 = 43 Milliarden
    \item Asymmetrischer Graph: Kantenlängen von einem zum anderen Konten sind nicht in beiden Richtungen identisch.\\
            Anzahl Routen: (n-1)!\\
            Bsp.:   15 Städte: (15-1)! = 87 Milliarden
\end{itemize}

\noindent Es gibt viele verschiedene Algorithmen, die versuchen, das \acs*{tsp} zu lösen.
Die exakten Algorithmen, welche eine optimale Lösung garantieren in dem sie beispielsweise jede mögliche Route berechnen, sind aber aufgrund der großen Anzahl an möglichen Routen nur für kleine Graphen möglich.


